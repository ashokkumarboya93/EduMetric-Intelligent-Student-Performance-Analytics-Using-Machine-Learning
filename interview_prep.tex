\documentclass[12pt,a4paper]{article}
\usepackage[utf8]{inputenc}
\usepackage{amsmath}
\usepackage{amsfonts}
\usepackage{amssymb}
\usepackage{graphicx}
\usepackage{hyperref}
\usepackage{listings}
\usepackage{xcolor}

\definecolor{codegreen}{rgb}{0,0.6,0}
\definecolor{codegray}{rgb}{0.5,0.5,0.5}
\definecolor{codepurple}{rgb}{0.58,0,0.82}
\definecolor{backcolour}{rgb}{0.95,0.95,0.92}

\lstset{
    backgroundcolor=\color{backcolour},   
    commentstyle=\color{codegreen},
    keywordstyle=\color{magenta},
    numberstyle=\tiny\color{codegray},
    stringstyle=\color{codepurple},
    basicstyle=\ttfamily\footnotesize,
    breakatwhitespace=false,         
    breaklines=true,                 
    captionpos=b,                    
    keepspaces=true,                 
    numbers=left,                    
    numbersep=5pt,                  
    showspaces=false,                
    showstringspaces=false,
    showtabs=false,                  
    tabsize=2
}

\title{EduMetric: Student Performance Analytics Platform \\ \large Interview Preparation Guide}
\author{Project Team}
\date{\today}

\begin{document}

\maketitle

\section{Project Overview}
\textbf{EduMetric} is an intelligent student analytics platform designed to predict academic outcomes and identify students at risk using Machine Learning (ML). Built with a Flask backend and a modern JavaScript frontend, the platform provides real-time insights into student performance, attendance, and behavioral trends.

\section{Key Features}
\begin{itemize}
    \item \textbf{Predictive Analytics}: Classified outcomes for Academic Performance, Risk Level, and Dropout Probability.
    \item \textbf{Interactive Dashboards}: Visualized insights using Plotly charts (Radar, Gauge, Donut, and Scatter plots).
    \item \textbf{Mentor Alert System}: Automated email notifications to mentors for students identified with "Poor" or "Medium" performance.
    \item \textbf{Cloud Persistence}: Scalable data management using Supabase (PostgreSQL).
    \item \textbf{CRUD Management}: Full student lifecycle management (Create, Read, Update, Delete).
\end{itemize}

\section{Detailed Execution Flow}
The system follows a synchronous flow from data entry to visual feedback:
\begin{enumerate}
    \item \textbf{Front-end Trigger}: The user inputs student details on the dashboard (e.g., Register Number, Semester Marks, Attendance).
    \item \textbf{API Request}: JavaScript (\texttt{app.js}) sends a POST request to the Flask server (\texttt{/api/student/predict}).
    \item \textbf{Feature Engineering}: The backend executes \texttt{compute_features()}, deriving statistical indicators like \textit{performance trend} and \textit{attendance percentage}.
    \item \textbf{ML Inference}: The processed features are passed to Gradient Boosting models loaded via \texttt{joblib}. These models return predicted labels (High/Medium/Low).
    \item \textbf{Data Storage}: Results are saved to Supabase for persistent tracking.
    \item \textbf{Responsive Visualization}: The frontend receives JSON data and dynamically renders various Plotly charts to illustrate the student's profile.
    \item \textbf{Automated Intervention}: If the prediction reveals a risk, the \texttt{api_send_alert} function is triggered to notify the mentor via SMTP.
\end{enumerate}

\section{Core Code Functionality}

\subsection{Feature Engineering (\texttt{app.py})}
The system calculates weighted scores to prepare data for the ML model.
\begin{lstlisting}[language=Python]
def compute_features(student):
    # Calculate Weighted performance score
    internal_pct = float(student.get("INTERNAL_MARKS", 0) or 0) / 30.0 * 100.0
    behavior_pct = float(student.get("BEHAVIOR_SCORE_10", 0) or 0) * 10.0
    
    # Statistical synthesis of attendance
    attendance_pct = present_att * 0.7 + prev_att * 0.2 + behavior_pct * 0.1
    
    # Final consolidated score for prediction
    performance_overall = past_avg * 0.5 + internal_pct * 0.3 + attendance_pct * 0.15 + behavior_pct * 0.05
    return { "performance_overall": performance_overall, ... }
\end{lstlisting}

\subsection{Machine Learning Prediction (\texttt{app.py})}
\begin{lstlisting}[language=Python]
def predict_student(features):
    X = np.array([
        features["past_avg"], features["past_count"], features["internal_pct"],
        features["attendance_pct"], features["behavior_pct"], features["performance_trend"]
    ]).reshape(1, -1)
    
    perf_pred = performance_model.predict(X)[0]
    # Inverse transform to get human-readable labels (e.g., 'High', 'Low')
    return {
        "performance_label": performance_encoder.inverse_transform([perf_pred])[0],
        ...
    }
\end{lstlisting}

\subsection{Frontend Analysis Flow (\texttt{app.js})}
\begin{lstlisting}[language=JavaScript]
async function analyseStudent(student) {
    showLoading("Analysing student...");
    const result = await api("/api/student/predict", "POST", student);
    if (result.success) {
        renderStudentHeader(result);
        renderStudentCharts(result); // Triggers Plotly.newPlot
        renderStudentSummary(result);
    }
}
\end{lstlisting}

\section{Interview Q\&A}

\subsection{What is the core problem EduMetric solves?}
It solves the problem of "delayed intervention" in education. Traditional systems only show past grades; EduMetric uses those grades along with behavioral and attendance data to \textit{predict} future outcomes, allowing teachers to help students \textit{before} they fail or drop out.

\subsection{Why did you use Supabase over local MySQL?}
Supabase provides a scalable, cloud-hosted PostgreSQL database with built-in real-time capabilities and easier deployment. This ensures that the student data is accessible from anywhere without complex server configurations.

\subsection{How does your system handle missing data?}
In the \texttt{compute_features} and \texttt{load_students_df} functions, we use Pandas' \texttt{fillna(0)} and Python's \texttt{get(key, default)} to ensure the application doesn't crash when optional fields are empty.

\subsection{What ML algorithms are used?}
The project uses Gradient Boosting models (Random Forest or XGBoost variants) because they handle tabular data exceptionally well and are robust against outliers in student performance metrics.

\subsection{How is the Risk Score calculated?}
The Risk Score is the inverse of the Performance Overall score. If a student's performance is 80\%, their risk score is approximately 20\%. Dropout risk is weighted more heavily toward attendance (70\%) as low attendance is a primary indicator of potential dropout.

\end{document}
